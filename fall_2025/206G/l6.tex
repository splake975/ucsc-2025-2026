\documentclass{article}
\usepackage{amsmath}
\usepackage{amssymb,bbm}
\usepackage{amsthm}
\usepackage{MnSymbol}
\usepackage{xcolor}
\usepackage{parskip}
\usepackage{tikz}
\usetikzlibrary{trees}


\theoremstyle{definition}
\newtheorem{definition}{Definition}
\newtheorem{theorem}{Theorem}


\newcommand{\norm}[1]{\lvert #1 \rvert}


\title{lecture}
\date{\today}
\begin{document}
\maketitle




\begin{definition}[NP-hard]
    every problem in NP can be efficiently reduced to any other problem in np hard
\end{definition}


max cut game:
each node is a player, each player decides if they wnat to move to the other side to increase the total score

this is equal to the local search for maxcut

\begin{definition}
    ppad complete (?), pls complete (?)
\end{definition}

\begin{definition}[pls class]
    has 3 algos that are efficient over input size.
    \begin{enumerate}
        \item initialization
        \item evaluation
        \item locally opt checker - report local opt or return better
    \end{enumerate}
\end{definition}

\begin{definition}[pls reduction]
    start from problem that is hard, transform into new algo in poly time, from \(\pi_1\) to \(\pi_2\).

    w requirements,
    \begin{enumerate}
        \item every \(x\in \pi_1\) maps to \(A(x)\in \pi_2\)
        \item maps every local optimum \(A(x)\) to \(x\) 
    \end{enumerate}






\end{definition}

if you can pls reduce max cut to a new problem and you claim to find local optimum in the new problem, then it would be a contradiction







\end{document}