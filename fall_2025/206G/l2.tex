\documentclass{article}
\usepackage{amsmath}
\usepackage{amssymb}
\usepackage{amsthm}
\usepackage{MnSymbol}
\usepackage{xcolor}



\theoremstyle{definition}
\newtheorem{definition}{Definition}
\newtheorem{theorem}{Theorem}


\newcommand{\abs}[1]{\left| #1 \right|}


\title{lecture}
\date{\today}
\begin{document}
\maketitle




payoff functions 
\begin{equation}
    u_i: S \rightarrow \mathbb{ R}
\end{equation}
\begin{itemize}
    \item \(I\) is the set of all players
    \item \(i\) is a player
    \item \(S_i\) is the set of all possible strategies for player \(i\)
    \item \(S\) is \textbf{space} strategy of profiles
    \item \(s\) is a specific strategy profile
    \item \(-i\) is defined as \(\{1,\dots,I\}\setminus i\)
    \begin{itemize}
        \item \(s_{-i} = (s_1,\dots , s_{i-1}, s_{i+1},\dots s_I)\)
    \end{itemize}
\end{itemize}

\begin{definition}[Nash Equilibrium]
    A strategy profile s from which no player has an incentive to deviate unilaterally
\end{definition}
ie for player \(i\)
\begin{equation}
    u_i(s_i,s_{-i})\geq u_i (s^\prime_i, s_{-i}) \forall s^\prime_i \in S_i
\end{equation}

it is generally not true that games have unique nash equilibria\\


\(r_i\) is the "Best response multifunction/correposndence"

\(r_i\) tells what the best response for a player is based on what they believe other players will do

note that given an opponent's strategy, the players strategy output of \(r_i\) might be multiple strategies, thus it is a multifunction rather than a function \footnote{a multifunction outputs a subset of the codomain. a correspondence is a function that outputs in \(2^D\) where D is the codomain (powerset of the codomiain)}

\begin{equation}
    r_i: S_{-i} \rightrightarrows s_i
\end{equation}

note: \(\rightrightarrows\) means that \(r_i\) is a \emph{multifunction}

\(\sigma_i\) is the probability distribution on \(S_i\)

\begin{equation}
    \sigma = \bigtimes_{i \in I} \sigma_i
\end{equation}


\(\Sigma_i\) is the space of all possible mixed strategies for player \(i\)

\(\)


\begin{equation}
    \Sigma = \bigtimes_{i \in I} \Sigma_i
\end{equation}

\textcolor{red}{\(\abs{s_i}-1\) simplex ??}




\(\sigma\) is a mixed nash equilibrium is \textcolor{red}{missing \(\forall\)delimiter}
\begin{equation}
    u_i(\sigma_i,\sigma_{-i})\geq u_i(\sigma^\prime_i,\sigma_{-i}),\forall\sigma_i \in \sigma
\end{equation}

\begin{equation}
    r_i: \Sigma_{-i}\rightrightarrows \Sigma_i
\end{equation}

\begin{equation}
    r_i(\sigma_-1)\mapsto \text{argmax}_{\sigma_i} u(\sigma_i,\sigma{-i})
\end{equation}

\begin{equation}
    r: \Sigma\rightrightarrows \Sigma
\end{equation}

\begin{theorem}[nash's theorem]
    In every finite game with a finite number of players and a finite set of pure strategies per player, there exists at least one Nash equilibrium in mixed strategies.
\end{theorem}


\begin{definition}[symmetric game]
    all players have same strategy space. a payoff to any player is the same for any other player i.e. permutations of any players strategies results in such that player i's strategy is played by player j results in the payoff of i and j being identical.
\end{definition}

\(S_i = S_j\), \(u_i(s_i,\dots,s_n) = u_j(s_{\pi(1)},\dots, s_{\pi(n)})\)

\begin{equation}
    \begin{split}
        S_i = S \quad \text{for all } i \\
        u_i(s) = u_j(\pi(s)) \quad \text{for any permutation } \pi \text{ of players}
    \end{split}
\end{equation}




\end{document}