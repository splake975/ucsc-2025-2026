\documentclass{article}
\usepackage{amsmath}
\usepackage{amssymb}
\usepackage{amsthm}
\usepackage{MnSymbol}
\usepackage{xcolor}
\usepackage{parskip}
\usepackage{tikz}
\usetikzlibrary{trees}


\theoremstyle{definition}
\newtheorem{definition}{Definition}
\newtheorem{theorem}{Theorem}


\newcommand{\norm}[1]{\lvert #1 \rvert}


\title{lecture}
\date{\today}
\begin{document}
\maketitle

\begin{definition}[1. NE]
    A strategy profile such that each player is playing a best response
\end{definition}

\begin{definition}[2. NE]
    A strategy profile and beliefs for each player about how each of the other players will play ST
    \begin{itemize}
        \item each player is maximizing payoff given those beliefs
        \item beliefs are correct
    \end{itemize}
\end{definition}


\begin{definition}[rational]
    player doesnt do things that give an inferior payoff
\end{definition}

if each player is rational and plays the rational response to the rational response to oinfinity, then it is noted that "rationality is common knowledge"

\begin{definition}[rationalizable profiles]
    profile + beliefs for each player:
    \begin{itemize}
        \item each player maximizes payoff wrt their beliefs
        \item beliefs are not inconsistent with rationality being common knowledge
    \end{itemize}
\end{definition}


\section{example: sealed first price auction}

consider p1 has valuation 3.5, p2 has valuation 1

assume each player knows the other players valuation

assume 1 is win, 0 i sloss

\begin{equation}
    P(i)=\begin{cases}
        i=1, \text{valuation}-\text{bid}\\
        i=0, 0
    \end{cases}
\end{equation}
if tie, \begin{equation}
    \frac{1}{2}(\text{valuation}-\text{bid})
\end{equation}


strongly/strictly dominated strategies are ones in which all possible outcomes for strategy a have less payoffs for strategy b 


a weakly dominated strategy is one in which all payouts are \(leq\) 


after removing strictly dominated strategies, you can reevaluate for new strictly dominated strategies


\begin{definition}[Iterated deletion of strictly dominated strategies]
    removing dominated strategies over and over until there are no more dominated strategies to remove. includes using mixed strategies
\end{definition}

\begin{equation}
    \text{all strategies} \subseteq \text{IDSDS} \subseteq \text{Rationalizable} \subseteq \text{NE}
\end{equation}

\section {ppng example}

\begin{equation}
    \begin{matrix}
        ~&~&&Kenney\\
        ~&~&&N&&&S\\
        Kimura&N&&0,0&&&10,-10\\
        ~&S&&5,-5&&&-1,1
    \end{matrix}
\end{equation}

if kenney goes north, then kimura goes south

if kenney goes south, then kimura goes north

if kimura goes north, kenney goes north

if kimura goes south, kenney goes south


therefore there is no pure strategy nash equilibrium

\textbf{IMPORTANT, solving}

p is probability kimura goes north, q is probability kenney goes north

kimura payoff N = payoff S 
if this is the case, then kenney is \emph{indifferent}

\begin{equation}\begin{split}
    q\cdot 0 = (1-q)\cdot 10 &= 5q + -1(1-q)\\
    q&=\frac{11}{16}
\end{split}
\end{equation}

kimuras indifference gives kenneys mix and vice versa. 


\begin{equation}
    \begin{split}
        0p + (-5)(1-p)&= -10p + (1-p)\\
        p=\frac{3}{8}\\
    \end{split}
\end{equation}



\section{battle of the people}

\begin{equation}
    \begin{matrix}
        ~&~&&bailey~&&&~&&&~\\
        ~&~&&F&&&o&&&D\\
        adrian&F&&3,2&&&1,1&&&-10,3\\
        ~&O&&0,0&&&2,3&&&-10,3\\
        ~&D&&-3,-10&&&-3,-10&&&-5,-5
    \end{matrix}
\end{equation}

3 pure strategy ne, f,f; o,o; d,d

hypothesis: there is a mixed NE w A's support fod, and B's support being fod


adrian prob: \(p_1,p_2,1-(p_1+p_2)\)

bailey prob: \(q_1,q_2,1-(q_1+q_2)\)

\begin{equation}\begin{split}
    3q_1+q_2-(1-q_1-q_2)&= 0+2q_2-10(1-q_1-q_2)\\
    &= -3q_1-3q_2-5(1-q_1-q_2)\\
\end{split}
\end{equation}
\begin{equation}
        q_1=\frac{5}{38}, q_2=\frac{15}{38}, q_3 = \frac{18}{38}
\end{equation}


if you get inconsistent or negative probabilities, then the strategies w those specific supports are not valid




\textbf{make all payoffs for player equal, find payoff based on opponents probability variables. set player profit equal, this find p2's probabilities}

\textbf{setting payoff equal for p1 gives probabilities for p2}











\end{document}