\documentclass{article}
\usepackage{amsmath}
\usepackage{amssymb,bbm}
\usepackage{amsthm}
\usepackage{MnSymbol}
\usepackage{xcolor}
\usepackage{parskip}
\usepackage{tikz}
\usetikzlibrary{trees}


\theoremstyle{definition}
\newtheorem{definition}{Definition}
\newtheorem{theorem}{Theorem}


\newcommand{\norm}[1]{\lvert #1 \rvert}


\title{lecture}
\date{\today}
\begin{document}
\maketitle

\section{matlab}

solving for modified battle of sexes from lec 4

set A,B matrix of payoffs,

\begin{equation}
    Ay=\text{vector of payoffs for each of adrians pure strategies}
\end{equation}

where y is baileys mix probabilities


for adrian to consider mixing across all strategies, \(Ay\) needs to equal a vector where each component is equal.

thus

\begin{equation}
    y=\frac{A^{-1} 1}{\norm{A^{-1} 1}}
\end{equation}

where \(1\) represents the ones vector


\(x^T\) is adrians mix, \textbf{\(x\) is transposed since representing the top player on the side requires transpose of the given matrix, thus more accurately it is described as \(B^Tx\)} where \(B\) is the apparent payoff matrix and \(x\) is adrians mix as a column
\begin{equation}
    x^TB
\end{equation}

\(x^TB\) is vector of expected payoffs of baileys strategies


for partial support strategies, remove the columns/rows that are not being played, and set y to zero on the non supported strategies. alternatively you can also just remove them from the matrix and mix vectors.

checking strategies outside of support is multiply 
\begin{equation}
    Ay
\end{equation}
and check if the payoff of the options outside of the support are greater or less than the supposed NE. 

\textbf{note:} the payoffs of the mixed strategies of the strategies minus the supposed support must be a linear combination of the pure strategies, thus only checking pure strategies is necessary

\textbf{question:} does negative, or inconsistent equations mean something? 

\section{systematize }

let \(A\) be a payoff matrix of a symmetric game.

assume that \(A \) is non-negative and \(n\times n\). (you can add constant matricies and outcomes dont change proof?)

consider the polytope 
\begin{equation}
    Az\leq \mathbbm{1}, z\geq 0
\end{equation}

note both of the inequalities have n entries

the bounds of these equations results in n tight inequalities, draws out a polytope


\begin{equation}
    Az\leq 1 \implies A\left(\frac{z}{\norm{z}}\leq \frac{1}{\norm{z}}\right)
\end{equation}


if this inequality is tight, i.e. the inequality becomes an equality, then it is a best response

\begin{definition}[represented strategies]
    strategy \(i\) is \emph{represented} at vertex \(z\) if either \(z_i=0\) or \(A_iz=1\)
\end{definition}


\textbf{observation: } if a vertex has all strategies represented, excluding the origin, then that vertex is a nash equilibrium

start at \(\vec 0=v_0\). pick an arbitrary strategy \(j\). look at vertecies adjacent to \(v_0\). in each of these verticies, exactly one positivity constraint gives slack and one constraint in \(Az=1\) is tight. therefore make positiviy constraint j, get slack and increase until something in \(Az\leq1\) tight. call the component that becomes tight, \(k\).

if \(k=j\), then j is the best response to itself and \(j\) is a pure equilibrium.

if \(k\neq j\), then \(k\) is doubly represented. \(z_k=0\),\([Az]_k=1\)

look at doubily represented strategy \(k\)

\begin{equation}
    (v_i)_k=0 
\end{equation}

and
\begin{equation}
    [Av_i]_k=1
\end{equation}

relaxing either of these, gives 2 new vertices

one of these has to be \(v_{i-1}\), go to the other vertex that wasnt just visited

\(Ax\leq b\), \(x\geq 0\),

\begin{equation}
    \begin{pmatrix}
        A|\mathbbm 1 
    \end{pmatrix}
    \begin{pmatrix}
        x\\
        s
    \end{pmatrix}
\end{equation}
where s is the slack variables. the first matrix is \(x\times 2n\) and the second \(2n\times 1\).








\end{document}