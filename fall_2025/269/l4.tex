\documentclass{article}
\usepackage{amsmath}
\usepackage{amssymb}
\usepackage{amsthm}
\usepackage{MnSymbol}
\usepackage{xcolor}
\usepackage{parskip}
\usepackage{tikz}
\usepackage{bbm}
\usetikzlibrary{trees}


\theoremstyle{definition}
\newtheorem{definition}{Definition}
\newtheorem{theorem}{Theorem}


\newcommand{\abs}[1]{\left| #1 \right|}


\title{lecture}
\date{\today}
\begin{document}
\maketitle

\section{Random Hyperplane Rounding (RHR)}

\begin{equation}
    \sum_{ij\in E}^{}w_{ij} \frac{\theta_ij}{\pi}
\end{equation}


optimal sdp value is 

\begin{equation}
    \max \sum_{ij\in E}^{} w_{ij}\left( \frac{1-\vec y_i\vec y_j}{2} \right)
\end{equation}


SDP value \(\geq\) OPT maxcut

\section{ hierarchical clustering HC}

\begin{equation}\label{1}
    \max \sum_{ij\in E}^{} w_{ij}\left( n-\abs{T_{ij}} \right)
\end{equation}

note \(n-\abs{T_{ij}}\) is the \# of non leaves

thus eq~\ref{1} can also be written as idk

levels of HC:
level \(t\) is a partition of \(V\), the verticies, in \(1,\dots, n\) into maximal clusters that have at most \(t\) nodes. 

if there is a \(t=0\), it is the same as \(t=1\).

\begin{definition}[maximal cluster]
    maximal clusters are such that merging any two clusters does not result in a valid clustering
\end{definition}


every t is considered a graph partitioning problem.

\begin{equation}
    x_{ij}^t=\begin{cases}
        0, &\text{at level $t$, $i,j$ are not separated}\\
        1, &\text{else}
    \end{cases}
\end{equation}


note eq~\ref{1} is equzl to

\begin{equation}
    \max \sum_{ij\in E}^{}w_{ij}\sum_{t=0}^{n}(1-x_{ij}^t)
\end{equation}


with constraints

\begin{equation}
    x_{ij}^{t+1}\leq x_{ij}^{t}
\end{equation}

and \begin{equation}
    \sum_{j\in V\setminus i}^{}x_{ij}^t\leq n-t
\end{equation}


this condition is called the spreading condition


for all \(i\in V\) assign \(\vec v_i^t\in \mathbb R^n\)

relaxing x into not just 0 and 1,

\begin{equation}
    x_{ij}^t = 1-v_i^tv_j^t
\end{equation}













\end{document}