\documentclass{article}
\usepackage{amsmath}
\usepackage{amssymb}
\usepackage{amsthm}
\usepackage{MnSymbol}
\usepackage{xcolor}
\usepackage{parskip}
\usepackage{tikz}
\usepackage{bbm}
\usetikzlibrary{trees}


\theoremstyle{definition}
\newtheorem{definition}{Definition}
\newtheorem{theorem}{Theorem}


\newcommand{\abs}[1]{\left| #1 \right|}


\title{lecture}
\date{\today}
\begin{document}
\maketitle

semi definite programming (?)

often try to take discrete problem, use continous relaxation then approxmate back into the discrete problem




continuous relaxation

\section{max cut algorithm}
max cut - maximize number of edges that cross between the two "sets"

\(G(V,E,W)\), \(w\in W>0\)

an algorithm \(A:V\rightarrow\{0,1\}\)

objective: maximize \[\sum_{uv\in E} w_{uv}\mathbbm{1} \{A(u)\neq A(v)\}\]

bipartite graph can be cut perfectly in half

new idea:
for every vertex \(v\in V\), variable \(x_v\in \{0,1\}\) tells whether the vertex \(v\) goes left or right. for every edge \(e\in E\), \(z_e\in \{0,1\}\) tells whether the edge is cut or not. 

objective function becomes:
\begin{equation}\label{2}
    \text{max}\sum_{uv\in E} w_{uv} Z_e
\end{equation}

w constraint: \begin{equation}\label{1}
    \begin{split}
        z_{uv}&\leq x_u+x_v\\
        z_{uv}&\leq 2-(x_u+x_v)
    \end{split}
\end{equation}
this is extremely cursed, use 
\(z_{uv} = x_u\oplus x_v\)?
\footnote{\(\oplus \) is xor}

nah 

approx is to use intervale \([0,1]\) for both x and z. \footnote{this breaks the xor notation}

\begin{definition}[LP]
    shorthand for linear program
\end{definition}

problem for usnig the interval, using~\ref{1},~\ref{2}, setting all \(x\) to 0.5, we can set \(z\) to 1 for everything and all programs give the max value again.

try another formulation since the last one doesnt work

\(\forall v\in V\), \(y\in \{-1,1\}\). use
\begin{equation}
    \text{max}\sum_{uv\in E} w_{uv}\frac{1-y_vy_u}{2}
\end{equation} 

try allow \(y_v\in \mathbb R^n\) where \(\abs{v}=n\), \(\abs{y_v} = 1\)

after placing the vertecies of \(y_u\) in space, taking random hyperplane to divide the set of vectors, we get something that is 0.878 optimal \footnote{???}


finding another algorithm that is better shows P=np ??????








\end{document}